%!TEX root = /Users/oroce/Documents/msc-szakdolgozat/dolgozat.tex
\begin{quotation}
A Continuous Integration - azaz a folyamatos integráció - egy olyan szoftverfejlesztési módszertan, amelyben a fejlesztőcsapat tagjai által írt kód napi rendszerességgel kerül (vagy automatizálással minden funkció, javítás implementálása után) integrálásra a korábbi fejlesztések közé. Minden új kód integrálása során automatizált tesztek ellenőrzik, hogy a rendszerbe való illesztés során okozott-e valamilyen hibát az új kódrészlet, illetve megfelel-e az adott fejlesztőcsapat által meghatározott minőségi kritériumoknak és ennek eredményéről a lehető leghamarabb visszajelzést ad. \cite{martin_fowler_cont_int}
\end{quotation} 

A szoftverfejlesztés során általában egy projekten több fejlesztő is dolgozik a kód különböző vagy akár egyazon részein is. A fejlesztők az alkalmazás másolatával dolgoznak a saját munkaállomásukon (nem pedig közvetlenül egy helyen). Ha egy feladat elkészül, akkor fel kell másolni a módosított fájlokat egy közös tárolóba/szerverre. Viszont a felmásolás előtt mindenképpen szükséges frissíteni a saját lokális példányukat, hogy elkerüljék a ütközéseket, illetve egymás munkájának felülírását, ezt a folyamatot nevezik integrációnak. Azonban előfordulhat, hogy a központi tároló és a saját lokális másolat között akkor különbség lehet, hogy nagy mértékben kénytelen a fejlesztő módosítani a saját fejlesztését, majd a javítás elvégzése után következhet ismét egy frissítés (integrálás), majd esetleg a megismételt javítás. Ez mint látható, egy ördögi kör. Ennek a megoldására jött létre a Continuous Integration, amelynek segítségével a fejlesztők adott időközönként megismétlik az integrációs lépést, hogy minél kevésbé térjen el a lokális verzió a központi tárolóban lévő verziótól.
\hfill\\
Hiába tűnik ez egy triviális és egyszerű megoldásnak, ez a megközelítés csak a 2000-es évek elején született meg, viszont azóta töretlen népszerűségnek örvend. Mára a folyamatos integráció összekapcsolódott az automatikus fordítással (build automation), azonban alapvetően nem szükséges része. Tehát például egy céges előírás, amely megköveteli, hogy a fejlesztők kötelesek minden reggel a lokális másolatuk frissítésére, is tulajdonképpen folyamatos integrációnak tekinthető, mert ezáltal megvalósítják a rendszeres integrációt. Ezek alapján kijelenthető, hogy a Continuous Integration valójában csak egy verziókezelő (pl.: git, Subversion, Mercurial) használatát követeli meg.
\\
Az automatizált fordítás habár nem alapvető része a folyamatos integrációnak mégis a fejlesztést és a fejlesztők munkáját nagyban megkönnyíti. Ez a művelet történhet bizonyos időközönként (munkaidő után éjszaka) vagy bizonyos eseményekre, mint például ha fejlesztők feltöltik a közös tárolóba a fájljaikat (commit). Az integrációs lépések és az automatikus fordítás közé érdemes lehet beépíteni a tesztelést, amely garantálja, hogy az új funkciók nem törik el a régi funkciókat. Továbbá érdemes az integrációról, a tesztelésről, és a fordításról riportokat generálni, melyekkel az eredmények - egy nem fejlesztő számára is - érthetőbb formába kerülnek.
\\
A folyamatos integrációra, tesztesetek futtatására, riportok értelmezésére már sok szoftver elérhető, többek között a Bamboo, amely a Jira és az Confluence mögött álló Atlassian terméke, vagy a Travic CI, amely ingyenes szolgáltatásként elérhető bármilyen nyílt forráskódú szoftver számára, illetve a mára de-facto rendszernek tekinthető Jenkins, amely a következő fejezetben kerül bemutatásra.

\section{Jenkins\\\small{https://jenkins-ci.org/}}
A Jenkins egy ingyenesen elérhető, nyílt forráskódú folytonos integráció támogató eszköz. A legtöbb verziókezelő rendszert támogatja és több mint 300 kiegészítő érhető el hozzá, melyek segítségével könnyedén testreszabható. Emellett támogatja az xUnit teszt keresztrenderek riportjait.

\section{TDD\\\small{Test Driven Development}}
Manapság a folyamatos integráció és az automatikus fordítás egyik legfontosabb velejáró kulcsszava a TDD, azaz a teszt vezérelt fejlesztés.
\\
A TDD használatához egy elég erős szemléletváltásra van szükség, ezért legalább annyi ellenzője van, mint támogatója. A teszt vezérelt vagy teszt irányított fejlesztés a nevével ellentétben nem egy tesztelési megoldás, hanem sokkal inkább tervezés. \cite{tddjs_definition}
\\
Egy alkalmazás jó és jól működéséhez könnyen bővíthetőnek kell lennie, hogy a termék, szolgáltatás meg tudjon újulni, a továbbfejlesztés zökkenőmentesen tudjon zajlani. De hogyan történik az új funkciók tervezése, beépíthetőségi megvalósításának tervezése?
\\
A TDD ezt próbálja elősegíteni azáltal, hogy a teszteseteket még a tervezés fázisában kell megírni, így a problémák a munka korai fázisában kiderülhetnek. A TDD teszt keretrendszerek általában könnyen olvashatóak még a fejlesztésben kevésbé járatos projekt résztvevők számára is, \aref{fig:mocha_should_tdd_test} ábra ezt kívánja bemutatni.
\ref{fig:mocha_should_tdd_test}
\begin{figure}[ht]
	\centering
		\lstinputlisting[language=Javascript]{assets/mocha_test.js}
		\caption{Egy TDD teszt mocha és should keretrendszerekkel}
		\label{fig:mocha_should_tdd_test}
\end{figure}
\\
Mivel a TDD teszteseteket a fejlesztés alatt folyamatosan futtatják - ellenben az egységtesztekkel, melyek általában az integráció során futnak le -, ezért kimondottan gyorsnak, mindenhol, bármilyen sorrendben futtathatónak kell lenniük, mert egyébként a fejlesztők nem fogják felhasználni őket. Teszteseteknek csak abban az esetben kellene változniuk, ha a kód ezt megköveteli, illetve ha a specifikáció változik.
\\
A fejlesztési folyamat négy lépésre bontható:
\hfill\\
\begin{description}
\item[Feladatok meghatározása:]\hfill\\
     Ebben a lépésben az ügyfél igényeknek megfelelő funkcionalitást kell feladat meghatározássá alakítani tesztesetek formájában. Meg kell határozni, hogy mit kellene tennie a rendszernek. Fontos, hogy nem azt kell itt kitalálni, hogy az adott fejlesztő hogyan valósítsa meg az adott feladatot, hanem hogy mit kell majd megvalósítani. A mit meghatározása által a fejlesztő is jobban megérti a feladatot, kisebb a félreértés lehetősége. A teszteset megírása után következik a megvalósítás.
\hfill\\
\item[Megvalósítás:]\hfill\\
     Ezután következik a hogyan valósítsuk meg az előre definiált feladatot. Azután meg is kell valósítani. A megvalósításnak kellene a legkönnyebb résznek lennie, ha mégsem könnyű akkor a következő problémák fordulhattak elő. \\
Túl nagy feladatot határoztunk meg az előző lépésben, így az a feladat, hogy kisebb feladatokra bontsuk. Felmerül a kérdés, hogyan lehet könnyű a megvalósítás, ha előtte még egy alkotóelemet is el kell előtte készíteni? Ilyenkor az adott alkotóelem feladatait kell meghatározni úgy, hogy a fejlesztők lesznek az ügyfelek és a ő igényeiket kell kielégíteni, és TDD alapján lefejleszteni.\\
Ha nem nagy lépésről van szó de mégis nehéz megvalósítani, akkor refaktorálni kell az adott részt. Ez lehet azért mert koszos a kód, vagy nincs jól megtervezve.
\hfill\\
\item[Ellenőrzés:]\hfill\\
     A megfelelő eszközzel le kell ellenőrizni, hogy sikeresek lettek-e a tesztek. Ennek gyorsnak és könnyűnek kell lennie. (A tesztek nem futhatnak néhány másodpercnél lassabban.) Ezáltal folyamatosan ellenőrzött, tervezett és fejlesztett lesz a kód.
\hfill\\
\item[Tisztítás:]\hfill\\
Ha sikeresen le lehet futtatni a teszteket, akkor következik a kód tisztítása. A működő kódot át kell nézi, a duplikációkat eltüntetni.\\ Beszédesebb neveket választhatunk a változóinknak. Mivel az ügyfél számára fontos funkcionalitás már le van tesztelve, ezért ez a művelet már könnyedén elvégezhető, ugyanis ha például egy metódus neve elgépelésre kerül, akkor rögtön jelez a teszt, hogy hiba van. Ettől tisztább és karbantarthatóbb lesz a kód.
\end{description}

Érdemes megfigyelni, hogy az összes lépés megfeleltethető a tervezés egy-egy részének. A meghatározott feladatok automatizálásának köszönhetően lehetséges, hogy kis biztonságos lépésekben történjen a rendszer felépítése és tervezése. Segíti a feladat megértését, rákényszerít a könnyű megvalósíthatóságra a folyamatos tisztítás és újratervezés segítségével.

\textbf{TDD előnyei:}
\hfill\\
\begin{itemize}
\item Refaktorálást segíti.
\item A kód módosítása könnyebb, hiszen hiba esetén a tesztek eltörnek, amely azonnali visszacsatolást a fejlesztő számára.
\item Könnyebb egy tesztelhető kódot refaktorálni (a tervezés miatt).
\item Az is segítség lehet, hogy hol nincs hiba.
\item Segít tesztelhetővé tenni az alkalmazást.
\item Rákényszerít, hogy ne legyen az alkalmazásban spagetti kód.
\item Felesleges funkció nem kerül megvalósításra, csak az ami a teszthez szükséges.
\item Előre kell tervezni.
\item Gyors, folyamatos visszajelzés kapható a funkció állapotáról (nem csak az adott fejlesztőknek, de a csapat többi tagjának és a projektmenedzsernek is).
\item Jobban ellenőrizhető a munka.
\item Van, hogy egy funkciónak nincs látható eredménye egy hétig. Ezzel szemben a TDD-nél naponta meg lehet mondani, hogy mennyi sikeres tesztet sikerült írni.
\item Segít megérteni a feladatot a példákon keresztül.
\item Időcsökkentő tényező a hibajavításnál és a refaktorálásnál.
\item Hibajavításnál segíthet pontosabban megjelölni a hiba helyét.
\item Akár dokumentációként is szolgálhat a teszt. Példakódnak tekinthető.
\item Biztosítja, hogy az új kód nem érint más tesztelt egységet.
\item Ha nincs TDD, akkor gyorsabban készül a szoftver, de nehezebben módosítható később.
\item A TDD tisztítás része akár kódfelülvizsgálatnak is tekinthető.
\item Stabilitást elősegítheti.
\end{itemize}

\textbf{TDD hátrányai:}
\hfill\\
\begin{itemize}
\item Nő a fejlesztési idő (refaktorálásnál csökken).
\item Ha nem tiszta, hogy mit kell tenni az adott feladattal könnyen előfordulhat, hogy rossz teszt kerül megírásra, amit át kell majd írni. (Újabb időnövelő tényező.)
\item Menet közben történt koncepcióváltásnál ki kell dobni a teszteket. (Újabb időnövelő tényező.)
\item A program működése nem lesz hibamentes, ha tesztek sikeres lefutnak.
\item Nem csodafegyver. A rendszer tesztelésének (minőségbiztosításának) csak egy kis részét kéne, hogy képezze. (Acceptance tesztelés, integrációs tesztelés mellett)
\item Csak tapasztalt fejlesztőkkel érdemes használni.
\item A tervezést nem mindig lehet úgy alakítani, hogy az megfeleljen a TDD-nek.
\item Hálózattal, fájlrendszerrel kapcsolatos dolgokra nem használható.
\item Páros programozásban a legjobb használni.
\item A tesztek írása unalmas lehet egyesek számára. Nagy fegyelemre van szükség.
\item Nehéz belerázódni, ezért arra a következtetésre lehet jutni, hogy semmi értelme.
\item TDD-ben tapasztalt párral lenne a legideálisabb.
\item Nehéz megmagyarázni a menedzsereknek/ügyfeleknek, hogy az elején miért tart ennyi ideig a fejlesztés.
\end{itemize}

%\section{QA\\\small{(Software) Quality Assurance}}
%Software quality assurance (SQA) consists of a means of monitoring the software engineering processes and methods used to ensure quality.[citation needed] The methods by which this is accomplished are many and varied, and may include ensuring conformance to one or more standards, such as ISO 9000 or a model such as CMMI.
%SQA encompasses the entire software development process, which includes processes such as requirements definition, software design, coding, source code control, code reviews, change management, configuration management, testing, release management, and product integration. SQA is organized into goals, commitments, abilities, activities, measurements, and verifications.[1]
%The American Society for Quality offers a Certified Software Quality Engineer (CSQE) certification with exams held a minimum of twice a year.

%http://www.inf.mit.bme.hu/sites/default/files/materials/category/kateg%C3%B3ria/oktat%C3%A1s/msc-t%C3%A1rgyak/szolg%C3%A1ltat%C3%A1sbiztons%C3%A1gra-tervez%C3%A9s-laborat%C3%B3rium/11/07_tesztautomatizalas_meresi_segedlet.pdf

%https://docs.google.com/viewer?a=v&q=cache:O4m0Cqx1xCIJ:www.inf.mit.bme.hu/sites/default/files/materials/category/kateg%25C3%25B3ria/oktat%25C3%25A1s/msc-t%25C3%25A1rgyak/szolg%25C3%25A1ltat%25C3%25A1sbiztons%25C3%25A1gra-tervez%25C3%25A9s-laborat%25C3%25B3rium/11/07_tesztautomatizalas_meresi_segedlet.pdf+&hl=hu&gl=hu&pid=bl&srcid=ADGEESikWw1iiMRvSQd9g3R6xUmTa5mMqznwpoWmxLZXxy8FQa1HhpwBSgBhLxvV6lDrk41WMoJwvwhWeMLTmyewJGvtTTA4g_hfn98BvkoSpMO0VLfzPY8ljCEyDpJWuAz-V9xbzij0&sig=AHIEtbTgfAo9T9Ti_kIhndMlokEW1PVDVQ
