A Continuous Integration egy olyan gyakorlat, melynek során a fejlesztői workspacek mergelése

Build server

TDD

QA

quality management

unit tests

“A Continuous Integration – azaz a folyamatos integráció – egy szoftver fejlesztési módszer melyben a fejlesztőcsapat tagjai az általuk írt kódot legalább napi rendszerességgel integrálják a korábbi fejlesztések közé, ez napi többszöri integrálást jelent. Minden új kód integrálása során automatizált tesztek ellenőrzik, hogy a rendszerbe való illesztés során okozott-e valamilyen hibát az új kódrészlet és ennek eredményeként a lehető leghamarabb visszajelzést ad az integráció eredményéről” [1][2].

http://www.inf.mit.bme.hu/sites/default/files/materials/category/kateg%C3%B3ria/oktat%C3%A1s/msc-t%C3%A1rgyak/szolg%C3%A1ltat%C3%A1sbiztons%C3%A1gra-tervez%C3%A9s-laborat%C3%B3rium/11/07_tesztautomatizalas_meresi_segedlet.pdf

https://docs.google.com/viewer?a=v&q=cache:O4m0Cqx1xCIJ:www.inf.mit.bme.hu/sites/default/files/materials/category/kateg%25C3%25B3ria/oktat%25C3%25A1s/msc-t%25C3%25A1rgyak/szolg%25C3%25A1ltat%25C3%25A1sbiztons%25C3%25A1gra-tervez%25C3%25A9s-laborat%25C3%25B3rium/11/07_tesztautomatizalas_meresi_segedlet.pdf+&hl=hu&gl=hu&pid=bl&srcid=ADGEESikWw1iiMRvSQd9g3R6xUmTa5mMqznwpoWmxLZXxy8FQa1HhpwBSgBhLxvV6lDrk41WMoJwvwhWeMLTmyewJGvtTTA4g_hfn98BvkoSpMO0VLfzPY8ljCEyDpJWuAz-V9xbzij0&sig=AHIEtbTgfAo9T9Ti_kIhndMlokEW1PVDVQ