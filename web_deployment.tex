%!TEX root = /Users/oroce/Documents/msc-szakdolgozat/dolgozat.tex
A webes alkalmazásoknál a folyamatos integráció több feladatot is ellát:
\begin{itemize}
	\item tesztek futtatása (unit, tdd, bdd, acceptance, integration)
	\item a fő verzióba való olvasztás
	\item adatbázis migrációs fájlok létrehozása
	\item kliensoldali statikus tartalmak konkatenálása, minimalizálása
\end{itemize}

A közösségi kódmegosztó github integrációs folyamata teljesen megfelel a felsorolásnak (\cite{github_deployment_web}). A különbség mindössze annyi, hogy miután sikeresek a continuous integration lépései egyből élesítésre kerülnek, azaz a fejlesztők a felelősek, ha valami hibát vétenek egy-egy funkció implementálásában.

Viszont a Facebook a saját PHP-ban írt alkalmazásának performancia javításának céljából további feladatokat is végez a folyamatos integrálás során, ez pedig a buildelés. A buildelés során a Facebook saját kódbázisát transzformálja amelynek köszönhetően hatszoros gyorsulást sikerült elérniük (a PHP kódot optimalizált C kódra alakítják, az átalakítás során használt szoftver \href{https://github.com/facebook/hiphop-php}{ingyenes elérhető}). \cite{facebook_deployment}
\subsubsection{Tesztek futtatása}
A tesztek futtatása a webes rendszerekben ugyanolyan fontos, mint bármelyik másik platformon. Viszont míg az asztali és mobil alkalmazásoknál, ismerhető a kliensek rendszer tulajdonságai (például ha egy alkalmazás csak Windows 7 operációs, akkor tudni lehet az azon elérhető futtathatósági lehetőségeket), azonban a webes alkalmazásoknál az nem csak különböző operációs rendszerekre kell optimalizálni, hanem az eltérő böngészőkre (melyek a lehető legkülönbözőbb módon implementálták az egyébként is laza HTML és ECMAScript szabványokat) és azok különböző verzióra. Ezeknek az okoknak köszönhetően a tesztelés, illetve a tesztautomatizálás rendkívül fontos a webes alkalmazásoknál (\cite{tddjs}).

\subsubsection{Kliensoldali tartalmak}
A kliensoldali tartalmak konkatenálása és minimalizálás rendkívül fontos webes alkalmazásoknál, ugyanis ezek a statikus tartalmak - a HTML mellett - az alkalmazás minden betöltésekor letöltésre kerülnek, ami pedig sok különálló fájl esetén a felhasználó élmény rovására mehet.
