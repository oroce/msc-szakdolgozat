%!TEX root = /Users/oroce/Documents/msc-szakdolgozat/dolgozat.tex
A folyamatos integrációnál mivel naponta több verzió is kikerülhet, sőt párhuzamosan akár több csapat vagy akár fejlesztő is élesíthet funkciókat, ezért nagyon fontos, hogy a rendszer folyamatosan tájékoztassa a csapatokat az élesített funkciók működéséről hibáiról illetve a többiek által fejlesztett funkciók állapotáról. Mindehhez két dologra van szükség, a kommunikáció és a rendszer minden apró részének monitorozására, mert ezáltal a hibák gyorsan kiszűrhetőek, a reakció idő csökkenthető és mindenki valós időben láthatja, hogy mi történik a rendszerben, azaz transzparens lesz, amelynek köszönhetően a szervezetnek könnyebb a részegységek megismerése, felelősségi köreik meghatározása, megismerése.

\section{Chat rendszer}
A fejlesztői csapatok illetve a csapattagok közti kommunikáció rendkívül fontos, mely persze történhet verbálisan, azonban ez megszakíthatja a munkát, illetve távoli munkavégzés esetén (főleg ha a résztvevők más országból végzik a munkát) akkor sokkal jobb megoldás egy chatrendszer bevezetése, ilyenek lehetnek például:
\begin{itemize}
	\item Jabber\nomenclature{Jabber}{TODO}
	\item IRC\nomenclature{IRC}{IRC - Internet Relay Protocol}
	\item HipChat
	\item Campfire
	\item Grove.io
	\item FlowDock
\end{itemize}

\begin{table}
	\caption{Chat rendszerek összehasonlítása}[h]
	\begin{tabular}{ c | c | c | c | c | c | c }
	- & Jabber & IRC & HipChat & Campfire & Grove.io & FlowDock \\
	\hline
			Saját szerver & \ding{51} & \ding{51} & \ding{55} & \ding{55} & \ding{55} & \ding{55} \\
			Mobil alkalmazás & \ding{51} & \ding{51} & \ding{51} & \ding{55} & \ding{51} & \ding{55} \\
			Asztali alkalmazás & \ding{51} & \ding{51} & \ding{51} & \ding{55} & \ding{55} & \ding{55} \\
			Egy-az-egy beszélgetés & \ding{51} & \ding{51} & \ding{51} & \ding{55} & \ding{55} & \ding{55} \\
			Csoportbeszélgetés & \ding{51} & \ding{51} & \ding{51} & \ding{55} & \ding{55} & \ding{55} \\
			Audió, Videó & \ding{51} & \ding{51} & \ding{51} & \ding{55} & \ding{55} & \ding{55} \\
			Email értesítés & \ding{51} & \ding{51} & \ding{51} & \ding{55} & \ding{55} & \ding{55} \\
	\hline
	\multicolumn{7}{l}{Forrás: \cite{chat_compare_hipchat}, \cite{chat_compare_campfire}, \cite{chat_compare_grove}, \cite{chat_compare_flowdock}}
	\end{tabular}
\end{table}
De miért kell chat, miért nem elég az email alapú kommunikáció?\\
\hfill\\
A legtöbb szervezetben sajnos még mindig az email a legfontosabb kommunikációs eszköz, azonban az email lassú, nem alkalmas instant üzenetek küldésére, illetve gyakran kimaradnak a levelezésből megfelelő emberek, ezzel ellentétben viszont a cseten gyorsan lehet üzenetet váltani. Természetesen a csettel nem lehet kiváltani az emailt, a két eszközt kombinálva, egymást segítve lehet jól használni.
Továbbá a cset mellett érvelve elmondható, hogy az emberek könnyebben elviselik a zajt csetet használva, mint az emailezésben.\textbf{}

\section{Naplózás, naplógyűjtés\\}

Az alkalmazások viselkedésének monitorozása rendkívül fontos, ugyanis amint kikerülnek a belső homokozóból, akkor megváltozhatnak a reakcióik a felhasználók különbőző tevékenységeire, ezeknek a problémák a felderítése használják a naplózás és a naplógyűjtést. Érdemes kiemelni, hogy a fejlesztési (development), tesztelési (staging/user acceptence testing) illetve az előnézeti (preview) szerverek is mind valamilyen homokozónak számítanak, hiszen valós felhasználók nem használják, a felhasználók szimulálása - akár tesztelők, akár automatizált tesztek által - szinte sosem tökéletesek.
\\
A jól használt naplózással a hibák hamarabb kiszűrhetőek, a hibák forrása könnyebben kiszűrhető, azonban minden rendszer naplójainak egyidejű figyelése egyszerűen lehetetlen. Ezért használnak a legtöbb rendszerben központi naplózást és naplógyűjtést, amelynek segítségével a hibák felfedezése, forrásának felderítése és a megoldása is könnyebben megoldható. Miért lehetne a központi naplózással könnyebben megoldani egy problémát? Amennyiben az alkalmazás példánya tegyük fel lassan válaszol, miközben a többi példány hiba nélkül működik (ezt egy könnyű ellenőrizni, hiszen egymás mellé helyezhető az alkalmazás példányok naplója), akkor egyértelmű, hogy a hiba a példányt futtató eszközön van csak jelen - a legtöbb esetben - amely megkönnyíti a szakembereknek a hibajelenség elhárítását.
\\
Továbbá a központi naplózás könnyen implementálható a folyamatos értesítés munkafolyamataiba, amellyel egy-egy alkalmazás/termék felelős szakembere azonnal értesülhet a fenálló hibáról, a javítást azonnal elkezdheti.
\\ 
A piacon több naplógyűjtő alkalmazás is elérhető mind telepíthető, mind \nomenclature{SaaS}{TODO} formában. A telepíthető és SaaS verzióban is elérhető többet a között a \emph{GrayLog2}.
\\
\subsection{GrayLog2}
A GrayLog2 segítségével könnyen monitorozható az összes szoftveres hiba, legyen szó akár adatbázisról, az operációs rendszer hibáiról vagy alkalmazáshibáról. Emelett lehetőséget nyújt nem csak hibajellegű üzenetek tárolására, hanem a hibakereséshez szükséges üzenetek megjelenítésére is.

\begin{figure}[ht]
	\centering
		\includegraphics[scale=0.5]{assets/graylog2.png}%
		\caption[DUMMY]%
		{A GrayLog2 interfésze}%
		\label{fig:graylog2-webinterface}
\end{figure}

\Aref{graylog2-webinterface} ábrán látható a GrayLog2 webes interfésze, amelynek segítségével a szakemberek könnyen észlelhetik valós időben a fennálló vagy éppen bekövetkező hibákat, továbbá kereshetnek a korábban előforduló hibákban (erre a GrayLog2 az \nomenclature{ElasticSearch}{TODO} \nomenclature{full-text search engine}{TODO} szoftvert használja) illetve könnyen beállíthatnak értesítéseket, hogy akár az éjszaka felmerülő hibákról is értesülhessen a felelős.

\section{Alkalmazás hiba monitorozás\\}
A naplózással a hibák könnyen észrevehetővé és később visszakereshető válnak, azonban a következőkre nem nyújt megoldást:\\
\begin{itemize}
\item hibák rögzítése a jegykezelő (issue kezelő) rendszerben
\item egy hiba csak egyszer kerüljön rögzítésre
\item automatikus értesítés a hibáról
\item a hibászázalék vizualizálása komponensenként
\end{itemize}

Természetesen alkalmazás hiba monitorozó rendszer nélkül is működhet rendszer jól, azonban előfordulhat, hogy nagy mennyiségű hibánál a naplógyűjtő rendszerben átsiklanak problémák felett vagy egyszerűen elfelejtésre kerülnek. A másik probléma lehet amikor a hibajelentést közvetlenül a jegykezelő rendszerbe kötik be. Ez miért lehet probléma? Egy nagy felhasználóbázissal rendelkező rendszernél, ha hiba csak a felhasználók néhány százalékánál fordul elő, akkor is szükségtelenül nagy mennyiségű jegy keletkezést fogja előidézni.\\

Az alkalmazás hiba monitorozás leginkább a nem webes alkalmazásoknál (mobil és asztali) terjedt el az utóbbi időben, ugyanis ezeknél a rendszereknél a naplók nem, vagy csak nehezen gyűjthetők valós időben.\\

Elérhető alkalmazások:
\begin{itemize}
\item Sentry - \url{http://getsentry.com}
\item Exceptional - \url{http://www.exceptional.io/}
\item Bugsense - \url{http://www.bugsense.com/}
\end{itemize}

\begin{figure}[ht]
	\centering
		\includegraphics[scale=1]{assets/sentry.png}%
		\caption[DUMMY]%
		{A Sentry interfésze}%
		\label{fig:sentry}
\end{figure}

\section{Szolgáltatások integrálása\\}
git commit hook
jenkins log
deploy
hibák
graylog
nagios
sentry
\subsection{Hubot}

GitHub, Inc., wrote the first version of Hubot to automate our company chat room. Hubot knew how to deploy the site, automate a lot of tasks, and be a source of fun in the company. Eventually he grew to become a formidable force in GitHub. But he led a private, messy life. So we rewrote him.

Today's version of Hubot is open source, written in CoffeeScript on Node.js, and easily deployed on platforms like Heroku. More importantly, Hubot is a standardized way to share scripts between everyone's robots.