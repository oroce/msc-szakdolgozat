%!TEX root = /Users/oroce/Documents/msc-szakdolgozat/dolgozat.tex
A szoftverek működtetése komplex dolog, de az adatalapú szemlélettel magabiztosan lehet őket működtetni. A magasfokú automatizálással a szoftverek külső beavatkozás nélkül tesztelhetőek, ellenőrizhetőek így csökkentve a hibázási lehetőséget. Az újrafelhasználással a komplex rendszerek is lebonthatóak apró alkotóelemekre, melyeknek megértése sokkal könnyebb és gyorsabb monolitikus társaiknál. Ez a két alkotóelem, amely meghatározza egy alkalmazás működtethetőségét technikai oldalról.\hfill\\
A másik oldalon viszont a szervezet tagjai vannak és a köztük történő kommunikáció. Az ismétlődő folyamatok automatizálásával csökken a szervezet tagjai felé támasztott humánerőforrás igény, viszont a tervezésnél és a problémák megoldásánál nagyfokú kommunikációra és koordinációra van szükség. Ennek a problémának a megoldása lehet a chat alapú, szobákra osztott, csoportos kommunikáció, amely az aszinkronitásával, az eszköz- és helyfüggetlenségével nagyban hozzájárulhat a szervezetek egy egységes, központi és transzparens információcseréjéhez. Továbbá a központi chat lehet az a gócpontja a szervezetnek, ahová befutnak a külső forrásokból érkező adatok, ahol egy-egy üzenetre az akció azonnal végrehajtható kontextus váltás nélkül, növelve a szervezet tagjainak információellátottságát, megteremtve a tudásmegosztás platformját, mindezt visszakereshetően és bárki számára elérhetően. Az incidensek felbukkanásakor a valósidejű viszont aszinkron kommunikáció a legjobb megoldás lehet arra, hogy a felhasználók észrevétlenül tudják továbbra is használni a szoftvert, miközben az érintettek maximális információellátottság mellett próbálják megoldani a felmerült problémát. Ha az incidensmenedzsmentbe bevonásra kerülnek az üzemeltetőkön kívül a fejlesztők is, akkor a tudásuk keresztezének köszönhetően a problémák forrásának detektálása és a megoldás ideje rövidülhet illetve a hibák utóéletének követése és azok elemzése - amelyeknek a működtetés részét kell képezniük - sokkal proaktívabbá válhat, és ezt a tudást felhasználva a szervezet többet profitálhat, a felhasználók elégedettebbek lesznek.

 %a mind üzemeltetési és mind fejlesztői tudással rendelkező szakemberek. A hibák utóéletének követése, azok elemzése a működtetés részének kell válnia és proaktívan kell segíteni a további fejlesztéseket.

% Mivel szoftverek üzemeltetésében a kommunikáció

%A dolgozat bemutatta, hogy miként lehet a kommunikáció ezen formáját a szoftverüzemeltetés kontextusában használni. 

%A dolgozat bemutatta ennek a valósidőben történő akcióvégrehajtásnak az előnyét a szoftveres hibák menedzselésére, miként lehet a fejlesztők és az üzemeltetők tudásánák keresztezését a szervezet javára fordítani, kitérve arra, hogy a hibautóélet kézelésének igenis a szoftverrendszerek részévé kell válnia és proaktívan kell segíteni a további fejlesztéseket.


 %Továbbá a szöveges üzenetekkel történő akciók elvégzése robot segítségével, amely a kontextus váltás csökkentése 

%Míg a szoftverek építésének folyamat optimalizációja, a megfelelő szoftverfejlesztési metodológiák használata a figyelem középpontjában vannak, addig maguknak a szoftverek működtetése kevés figyelmet kap. 
%A dolgozat bemutatja a webes alkalmazásokra fókuszálva, hogy egy alkalmazás életútja során egészen a kód megírásától a tesztelésen, az élesítésen át a hibaelhárításig milyen lépéseket kell követni, az iparágban leggyakrabban használt eszközök segítségével. \hfill\\

%Majd bemutatásra kerül, hogy a felhasznált eszközök hogyan integrálhatóak egybe, miként tudja a szervezet kihasználni a központosított aszinkron kommunikáció előnyeit, csökkentve a kontextusváltások és a folyamatos megszakítások okozta stresszt. 