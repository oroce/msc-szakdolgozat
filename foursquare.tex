\section{Bevezetés}

A foursquare egy helymegosztás alapú közösségi oldal, 2009 óta működik. A foursquare segítségével a városok gyorsabban felfedezhetőek, felhasználók által közkedvelt helyek hamarabb megtalálhatóak. Az oldal működésének elve, hogy ha a felhasználó elmegy egy étterembe, akkor be tud jelentkezni, hogy az adott helyen van, ilyenkor kaphat az étteremben kedvezményt illetve a barátai láthatják, hogy jelenleg hol tartózkodik. Mint a leírásból is kiderül az oldal alapvetően mobileszközökön használható jól, mert az aktuális földrajzi helyzethez képest ajánlja fel a rendszer helyeket. A helyzetmeghatározás történhet GPS pozíció, GSM tornyok beháromszögelése és a W3C\nomenclature{W3C (World Wide Consortium)}{\hfill\\Egy konzorcium, amelynek a webes technológiák fejlesztésének irányítása, és technológiai iránymutatás (ajánlások) a célja.} HTML5\nomenclature{HTML5}{\hfill\\A HTML új verziója, melybe már olyan megoldások kerültek bele, mint a geolokalizáció, az animációk támogatása, böngészőben futó adatbázisok, 3D modellezés.} draftjában található geolokalizáció alapján is. (\defcitealias{foursquarecom}{foursquare.com, 2011}\citetalias{foursquarecom})

A foursquare-t azért választottam bemutatandó példának, mert dokumentum-orientált adatbázist használ, és az egyik legrégebbi MongoDB-t használó, komoly felhasználóbázissal rendelkező szolgáltatás.
A közösségi oldal 2011. áprilisában mintegy 8 millió felhasználóval rendelkezett és ez a szám naponta körülbelül 35 ezerrel növekszik. A felhasználók naponta körülbelül 2,5 milliószor jelentkeznek be (az előző évben megközelítőleg fél milliárd bejelentkezés történt), ennek ellenére a szolgáltatás mögött álló foursquare Labs, Inc. mindössze 60 alkalmazottal rendelkezik. (\defcitealias{foursquarecom_numbers}{foursquare.com, 2011b}\citetalias{foursquarecom_numbers})
\\
Az alkalmazás az Amazon EC2-ben\footnote{az induló alkalmazások (startup) egyik legkedveltebb felhőszolgáltatója} elhelyezett 25 darab szerveren fut, scalaban\footnote{Java és ML alapú programozási nyelv, Java virtuális gépen (JVM) fut} írva.
A rendszer indulásakor PostgreSQL adatbázist használt, azonban a felhasználók és bejelentkezések számának növekedésével a fejlesztők kénytelenek voltak az adatbázis réteg cseréje mellett dönteni, mivel a relációs rendszer skálázása (mind a szétosztás, mind a replikáció) komoly problémát jelentett. A választás a MongoDB-re esett, amelyhez a szerverek átalakítására volt szükség, ugyanis a NoSQL adatbázisok, a lemezre írás és lemezről olvasás helyett, a memóriát használják, ezzel csökkentve a lekérdezések idejét. Azonban a PostgreSQL a mai napig használatban van (az adatok rögzítése nem csak a MongoDB-be kerül írásra, hanem a relációs adatbázisba is), tehát egyfajta biztonsági mentésként üzemel.

% és ellentétben a legtöbb NoSQL adatbázist használó rendszerrel, a foursquare nem relációs adatbázisról váltott, hanem már a rendszertervezésnél figyelembe vehették a dokumentum-orientált adatbázis nyújtotta előnyöket.

\section{MongoDB a foursquare-nél}

Ahogyan az előző pontban olvasható, a szervereket át kellett alakítani a MongoDB-hez, ez annyit jelent, hogy bővítették a memóriát 66 gigabájtig, amellyel elérték, hogy a legtöbb információ eléréséhez nem kell, a merevlemezről olvasni, ugyanis elfér a memóriában.


\subsection{Előnyök}

	\paragraph{Skálázhatóság}
		A foursquare - mint jelenleg feltörekvő szolgáltatás - esetében a folyamatos felhasználónövekedés miatt, folyamatosan szükséges a szerverkapacitás bővítése. A hardveres bővítést az Amazon oldja meg, míg az új adatbázis-kezelő bekapcsolása a szolgáltatásba a foursquare dolga, amely szerencsére nem bonyolult MongoDB segítségével, mindössze be kell állítani az elosztást végző szerveren az új eszköz paramétereit, és onnantól kezdve az adatok szétosztása és lekérdezése automatikusan működik.

	\paragraph{Geolokalizáció}
		Mivel a szolgáltatás alapja a lokalizált találat, és a MongoDB-ben megtalálható földrajzi koordináták indexelése, ezért a legközelebbi helyek megtalálásának és listázásának rövid válaszidejét nagyban elősegíti az adatbázisnak ez a funkciója. (\defcitealias{harryheymannprez}{Foursquare \& MongoDB, 2010a}\citetalias{harryheymannprez})
 
 	\paragraph{Rövid válaszidők}
 	A rövid válaszidőknek köszönhetően az alkalmazás több kérést tud indítani az adatbázis felé, amely a foursquare esetében 5000(!) kérés másodpercenként, mindezt 25 szerver segítségével oldják meg. (\defcitealias{harryheymannqa}{Q and A from Scaling foursquare with MongoDB, 2010}\citetalias{harryheymannqa})
\subsection{Hátrányok} 

	\paragraph{Indexelt mezők méretének limitálása}
		Az indexelt mezők korlátja egy kilobájt, a foursquare-nél ez komoly problémákat okoz, ugyanis az egy-egy helyhez több tíz vagy akár száz kulcsszó is tartozhat. Az indexelés limitálásával a kulcsszavak egy része lesz csak felindexelve, míg a többi kulcsszóra a keresés lehetetlenné válik. A foursquare a problémát a sok kulcsszóval rendelkező helyeknél, az indexelés kihagyásával kerülte meg. (\defcitealias{harryheymannvid}{Foursquare \& MongoDB, 2010b}\citetalias{harryheymannvid})
