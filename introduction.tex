Az interneten böngészve a felhasználó gyakran észre sem veszi, de egy weboldalon történő kattintása között elképzelhető, hogy az adott rendszer kétszer is verziót váltott, azaz frissítette a rendszerét. De hogyan történik mindez? Hogyan lehetséges leállás nélkül naponta többször verziót váltani, akár földrajzilag elosztott rendszereket populálni? Ezekre a kérdésekre kíván a szakdolgozat válaszokat adni.
\\
Bemutatásra kerül a continuous integration folyamat, ezekhez használt rendszerek, eszközök.
\\
A dolgozat témája alapvetően a webes alkalmazásokat hivatott bemutatni, azonban kitér az asztali alkalmazásokra is, illetve figyelmet szentel a mobil alkalmazásoknak is (mind az iOS, Android és a Windows Phone platformoknak).