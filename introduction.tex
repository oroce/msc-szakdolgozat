%!TEX root = /Users/oroce/Documents/msc-szakdolgozat/dolgozat.tex
Az interneten böngészve a felhasználó gyakran észre sem veszi, hogy egy weboldalon történő két kattintása között előfordulhat, hogy az éppen böngészett weboldal mögötti szoftverben verzióváltás történt. De hogyan történik mindez? Hogyan lehetséges leállás nélkül naponta többször verziót váltani, akár földrajzilag elosztott rendszereket kezelni? Ezekre a kérdésekre kíván a szakdolgozat válaszokat adni.
\\
Bemutatásra kerül, hogy hogyan lehetséges a kód minőségének folyamatos ellenőrzése és annak integrálása, hogyan tudja a szervezet kezelni és megszervezni saját kommunikációját annak érdekében, hogy a szoftverrendszerek fejlesztése folyamatos legyen.
\\
Míg maguk a szoftverfejlesztési metodológiák az elmúlt évek egyik kedvenc témája, addig a szoftverrendszerek működtetése egy kevésbé a figyelem központjában lévő szakmai feladat. A dolgozat nem kívánja a működtetés minden apró mozzantát bemutatni, hanem egy résztémakör fontos lépéseire próbál fókuszálni, felhívva a figyelmet a meghatározó sarokpontokra. Továbbá próbálja egy olyan megoldást feldolgozva bemutatni a kódkészítés-ellenőrzés-beolvasztás-élesítés-hibadetektálás-javítás folyamatát, melynek használatával és segítségével a tanuló szervezet reszponzívabb és átláthatóbb lehet.
\\
A dolgozat alapvetően a webes alkalmazásokra fókuszál, de technológiafüggetlen, és az itt leírt pontok akár asztali- vagy mobilalkalmazásoknál is felhasználhatóak, sőt a dolgozatban felsorakoztatott lépéseik létező projektek tapasztalataiból épülnek fel. A szerző közel 7 éve dolgozik a szoftverfejlesztés és szoftverüzemeltetés területén, mint fejlesztő, fejlesztési- és projektvezető. A dolgozat megpróbál rávilágítani azokra a szerző által problémásnak vélt szervezeti, üzemeltetési és szoftverfejlesztési problémákra, amelyek javításával a szoftverműködtetés a szervezet gördülékeny, elemi része lehet.