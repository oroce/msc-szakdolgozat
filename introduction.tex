%!TEX root = /Users/oroce/Documents/msc-szakdolgozat/dolgozat.tex
Az interneten böngészve a felhasználó gyakran észre sem veszi, de egy weboldalon történő két kattintása között elképzelhető, hogy az éppen böngészett rendszer verziót váltott, azaz frissítette a rendszerét. De hogyan történik mindez? Hogyan lehetséges leállás nélkül naponta többször verziót váltani, akár földrajzilag elosztott rendszereket populálni? Ezekre a kérdésekre kíván a szakdolgozat válaszokat adni.
\\
Bemutatásra kerül, hogy hogyan lehetséges a kód minőségének folyamatos ellenőrzése és annak integrálása. Hogyan tudja a szervezet kezelni és megszervezni saját kommunikációját annak érdekében a szoftverrendszerek fejlesztése folyamatos legyen.
\\
Míg maga a szoftverfejlesztési metodológiák az elmúlt évek egyik kedvenc témája, addig a szoftverrendszerek működtetése egy kevésbé a figyelem központjában lévő szakmai feladat. A dolgozat nem kívánja a működtetés minden apró mozzantát bemutatni, hanem egy résztémakör fontos pontjaira próbál fókuszálni, felhívva a figyelmet a meghatározó sarokpontokra. Továbbá próbálja egy olyan megoldást feldolgozva bemutatni a kódkészítés-ellenőrzés-beolvasztás-élesítés-hibadetektálás-javítás folyamatát, melynek használatával és segítségével a tanuló szervezet reszponzívabb és átláthatóbb lehet.
\\
A dolgozat alapvetően a webes alkalmazásokra fókuszál, de technológiafüggetlen, és az itt leírt pontok akár asztali- vagy mobilalkalmazásoknál is felhasználható.