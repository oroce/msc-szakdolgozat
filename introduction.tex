%!TEX root = /Users/oroce/Documents/msc-szakdolgozat/dolgozat.tex
Az interneten böngészve a felhasználó gyakran észre sem veszi, de egy weboldalon történő két kattintása között elképzelhető, hogy az adott rendszer kétszer is verziót váltott, azaz frissítette a rendszerét. De hogyan történik mindez? Hogyan lehetséges leállás nélkül naponta többször verziót váltani, akár földrajzilag elosztott rendszereket populálni? Ezekre a kérdésekre kíván a szakdolgozat válaszokat adni.
\\
Bemutatásra kerül a continuous integration folyamat, ezekhez használt rendszerek, eszközök.
\\
A dolgozat témája alapvetően a webes alkalmazásokat hivatott bemutatni, azonban kitér az asztali alkalmazásokra is.
\\
A verzióváltás téma azért rendkívül izgalmas, mert akkor működnek jól, ha a végfelhasználó nem veszi észre, azonban ezeknek a rendszerek megtervezése, kiépítése, működtetése néha komolyabb mérnöki probléma, mint azok az alkalmazások, amelyek számára készültek.
